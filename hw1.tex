\documentclass[11pt]{article}
\usepackage{geometry} % See geometry.pdf to learn the layout options. There are lots.
\geometry{letterpaper}
\usepackage{graphicx}
\usepackage{amsmath}
\usepackage{amssymb}
\usepackage{fontspec}
\usepackage{newunicodechar}
\newunicodechar{Ⓗ}{\hcircle}

\usepackage{url}
\usepackage[utf8]{inputenc} 
% \usepackage{epstopdf}
\DeclareGraphicsRule{.tif}{png}{.png}{`convert #1 `dirname #1`/`basename #1 .tif`.png}

\title{CMSI 284 Encoding Exercise\\
or, the Joy of Hex}
\author{Submitted by: Harris Lummis}
\date{}

% found command at http://tex.stackexchange.com/questions/131125/better-way-to-display-long-division
\newcommand\Mydiv[2]{%
$\strut#1$\kern.25em\smash{\raise.3ex\hbox{$\big)$}}$\mkern-8mu
        \overline{\enspace\strut#2}$}
        
\begin{document}
\maketitle


\section*{Instructions}

Do all of these problems \emph{without} the aid of a computer. The purpose of these exercises
is for you to develop ``manual'' encoding skills, which you will need in the event that a
zombie apocalypse wipes out all known systems technology on the planet.

You may submit this assignment in one of these ways. With \emph{both} options, make sure to
show your work where work needed to be done. This provides evidence that you did not use a
computer to determine the answers.

\begin{itemize}
\item If you know \LaTeX\ sufficiently, copy the \emph{source file} of this exercise and add your
solutions to this copy. Commit and push the file to your GitHub repository. \emph{Advantage:}
Drop-dead clear, sharp, unambiguous presentation. \emph{Disadvantage:} Intermediate computations
may be harder to write down.

\item Alternatively, you may \emph{print} the PDF version of this exercise and do your work on
paper. Submit this printout with your \emph{name} in the designated blank up top. \emph{Advantage:}
More convenient for showing your work. \emph{Disadvantage:} Handwritten answers may be harder to
read.
\end{itemize}


\section*{Mapping to Outcomes and Proficiencies}

The overall assignment covers outcomes \(1a\), \(1b\), \(4d\), and \(4f\). Each question will
pertain specifically to either \(1a\) or \(1b\) and will be given a score ranging from 0 to 4
based on the correctness of the answer. The average score for a given outcome, rounded,
determines the final proficiency for the assignment. e.g., If your numeric encoding answers
attain an average of 3.2, then \(1a\) will get a proficiency of \texttt{|}.

Outcome \(4d\) will be determined by how well you use the information given in class to compute
the requested answers, and how accurately you follow the instructions in this assignment.

Outcome \(4f\) will be determined by whether you submit the assignment on time.

\pagebreak

\section{Integers}

\emph{Outcome 1a:} Assuming a 16-bit storage word, choose a value in the requested encoding and
specification, then provide its corresponding values for the other encodings:
\linebreak \linebreak For the following problems, these powers of 2 may come in handy:
\begin{tabular}{ l | r }
	x & $2^{x}$ \\
	\hline
	10 & 1024 \\
	11 & 2048 \\
	12 & 4096 \\
	13 & 8192 \\
	14 & 16384 \\
	15 & 32768 \\
	16 & 65536
\end{tabular}
\begin{enumerate}

\item Signed decimal \(< -31000\), not divisible by 2, 4, or 8:

-31567

\begin{enumerate}
\item \text{Unsigned decimal}: \linebreak\linebreak
\text{1000 0100 1011 0001}\linebreak\linebreak
\begin{align*}
1 \times  2 \times 2 \times 2 &= 8 \\
(8 \times 2 \times 2 + 1) \times 2 \times 2 &= 132\\
132 \times 2 + 1 &= 265\\
265 \times 2 &= 530\\
530 \times 2 + 1 &= 1061\\
1061 \times 2 + 1 &= 2123\\
2123 \times 2 &= 4246\\
4246 \times 2 &= 8492\\
8492 \times 2 &=  16984 \\
16984 \times 2 + 1 &= \boxed{33969}
\end{align*}
\item Hexadecimal: \linebreak\linebreak
Use a hex digit to represent each four digit block of signed binary integer \linebreak\linebreak
\begin{tabular}{c c c c}
1000 & 0100 & 1011 & 0001 \\
8 & 4 & B & 1
\end{tabular} \linebreak\linebreak
\boxed{84B1} \pagebreak
\item Binary: 
\linebreak
\linebreak
\begin{tabular}{ l  l}
	\begin{tabular}{ r | l }
		\Mydiv{2}{31567} & 1\\
		\Mydiv{2}{15783} & 1\\
		\Mydiv{2}{7891} & 1\\
		\Mydiv{2}{3945} & 1\\
		\Mydiv{2}{1972} & 0\\
		\Mydiv{2}{986} & 0\\
		\Mydiv{2}{493} & 1\\
		\Mydiv{2}{246} & 0\\
		\Mydiv{2}{123} & 1\\
		\Mydiv{2}{61} & 1\\
		\Mydiv{2}{30} & 0\\
		\Mydiv{2}{15} & 1\\
		\Mydiv{2}{7} & 1\\
		\Mydiv{2}{3} & 1\\
		\Mydiv{2}{1} & 1 \\
	\end{tabular} &
	\begin{tabular}{l}
		0111 1011 0100 1111 \\ 
		Flip all bits after rightmost 1 \\
		\boxed{\text{1000 0100 1011 0001}}
	\end{tabular}
	
\end{tabular}
\end{enumerate}

\item Hexadecimal between \texttt{A234} and \texttt{DFFF} inclusive, no zeroes:\linebreak

CAFE

\begin{enumerate}
\item Unsigned decimal:
\begin{align*}
	1\times 2 + 1 &= 3\\
	3\times 2\times 2 &= 12\\
	12\times2+1\times2&=50\\
	50\times2+1\times2&=202\\
	202\times2+1 &= 405\\
	405\times2+1 &= 811\\
	811\times2+1&=1623\\
	1623\times2+1&=3247\\
	3247\times2+1&=6495\\
	6495\times2+1&=12991\\
	12991\times2+1&= 25983\\
	25983\times2&=\boxed{51966}
\end{align*}
\item \text{Signed decimal:}\linebreak
\text{1100 1010 1111 1110}\linebreak
\text{First bit is 1, so number is negative. Flip all bits after first 1.}\linebreak
\text{0011 0101 0000 0010}
\begin{align*}
	1\times2+1 &= 3\\
	3\times2&=6\\
	6\times2+1&=13\\
	13\times2&=26\\
	26\times2+1&=53\\
	53\times2&=106\\
	106\times2&=212\\
	212\times2&=424\\
	424\times2&=848\\
	848\times2&=1696\\
	1696\times2&=3392\\
	3392\times2+1&=6785\\
	6785\times2&=13570\\
	-1\times13570&=\boxed{-13570}
\end{align*}
\item Binary: \linebreak
\begin{tabular}{c c c c}
C & A & F & E \\
1100 & 1010 & 1111 & 1110
\end{tabular}\linebreak
\boxed{\text{1100 1010 1111 1110}}
\end{enumerate}

\item Hexadecimal between \texttt{0111} and \texttt{01FF} inclusive, two zeroes max:

01E0

\begin{enumerate}
\item Unsigned decimal:

	The number is positive in a signed representation, so its unsigned representation will be equivalent. \boxed{480}\pagebreak
	
\item \text{Signed decimal:}

	Using binary representation, first digit is a 0 so number is positive.
	\begin{align*}
		1\times 2^8 &= 256\\
		256+2^7 &= 384\\
		384+2^6 &= 448\\
		448+2^5 &= \boxed{480}		
	\end{align*}
\item Binary:\linebreak
	Encode each Hex digit using 4 bits:
		\begin{tabular}{c c c c}
			0 & 1 & E & 0\\
			0000 & 0001 & 1110 & 0000
		\end{tabular}
		
	\boxed{\text{0000 0001 1110 0000}}

\end{enumerate}

\item Binary with high-order bits \texttt{1011} and at least 5 \texttt{1}s:

1011 0111

\begin{enumerate}
\item Unsigned decimal:

1011 0111

\begin{align*}
1 \times 2^7 &= 128\\
128 + 2^5&=160\\
160+2^4&=176\\
176+2^2&=180\\
180 + 2^1 &= 182\\
182+2^0 &= \boxed{183}
\end{align*}
\item Signed decimal:

First bit is a 1, so number is negative. Flip all bits to the left of the rightmost 1:

\begin{tabular}{c c}
1011 & 0111\\
0100 & 1001
\end{tabular}

\begin{align*}
1\times 2^6 &=64\\
64+2^3&=72\\
72+2^0&= \boxed{73}
\end{align*}
\item Hexadecimal:

One hex digit represents each 4 binary digits:
\begin{tabular}{c c}
1011 & 0111\\
B & 7
\end{tabular}

\boxed{B7}
\end{enumerate}

\item Hexadecimal between \texttt{8000} and \texttt{A000} \emph{exclusive}, one zero max:

890B

\begin{enumerate}
\item Unsigned decimal:
\begin{align*}
8\times16^3&=8\times4096\\
8\times4096&=32768\\
32768 + 9\times256&=32768+2304\\
32768+2304&=35072\\
35072+11&=\boxed{35083}
\end{align*}
\item Signed decimal:

First digit is a 1, so number is negative. Flip all bits left of rightmost 1.

\begin{tabular}{c c c c}
1000&1001&0000&1011\\
0111&0110&1111&0101
\end{tabular}

\begin{align*}
	(1\times2+1)\times2+1&=7\\
	7\times2\times2+1&=29\\
	(29\times2+1)\times2&=118\\
	(118\times2+1)\times2+1&=475\\
	475\times2+1 &= 951\\
	951\times2+1 &= 1903\\
	1903\times2 &= 3806\\
	3806\times2+1 &= 7613\\
	7613\times2 &= 15226\\
	15226\times2+1 &= 30453\\
	\boxed{-30453}
\end{align*}
\item Binary: 

Use 4 bits to represent each hex digit


\begin{tabular}{c c c c}
8&9&0&B\\
1000&1001&0000&1011
\end{tabular}

\boxed{\text{1000 1001 0000 1011}}
\end{enumerate}

\item Unsigned decimal between 48000 and 65000 inclusive, not divisible by 4 or 8:

48391

\begin{enumerate}
\item Signed decimal:

\text{1011 1101 0000 0111}

First bit is a 1, number is negative. Flip all bits left of rightmost 1.

\text{1011 1101 0000 0111}


\text{0100 0010 1111 1001}

\begin{align*}
	1\times2^{14}&=16384\\
	16384 + 2^9&=16896\\
	16896 + 2^7&=17024\\
	17024 + 2^6&=17088\\
	17088 + 2^5&=17120\\
	17120 + 2^4&=17136\\
	17136 + 2^3&=17144\\
	17144 + 2^0&=17145\\
	&\boxed{-17145}
\end{align*}
\item Hexadecimal:

Use a hex digit to represent each nybble:

\begin{tabular}{c c c c}
1011&1101&0000&0111\\
B&D&0&7
\end{tabular}

\boxed{BD07}

\item Binary:

\begin{tabular}{r l}
\Mydiv{2}{48391}&1\\
\Mydiv{2}{24195}&1\\
\Mydiv{2}{12097}&1\\
\Mydiv{2}{6048}&0\\
\Mydiv{2}{3024}&0\\
\Mydiv{2}{1512}&0\\
\Mydiv{2}{756}&0\\
\Mydiv{2}{378}&0\\
\Mydiv{2}{189}&1\\
\Mydiv{2}{94}&0\\
\Mydiv{2}{47}&1\\
\Mydiv{2}{23}&1\\
\Mydiv{2}{11}&1\\
\Mydiv{2}{5}&1\\
\Mydiv{2}{2}&0\\
\Mydiv{2}{1}&1\\
\end{tabular}\linebreak

\boxed{\text{1011 1101 0000 0111}}
\end{enumerate}

\item Unsigned decimal between 80 and 1024 \emph{exclusive}, not divisible by 4 or 8:

93

\begin{enumerate}
\item Signed decimal:

First bit is a zero, so number is positive, thus unsigned and signed representations are identical in decimal value. \boxed{93}
\item Hexadecimal:

Use a hex digit to represent each nybble:

\begin{tabular}{c c c c}
0000&0000&0101&1101\\
0&0&5&D
\end{tabular}

\boxed{005D}
\pagebreak
\item Binary:

\begin{tabular}{r l}
\Mydiv{2}{93}&1\\
\Mydiv{2}{46}&0\\
\Mydiv{2}{23}&1\\
\Mydiv{2}{11}&1\\
\Mydiv{2}{5}&1\\
\Mydiv{2}{2}&0\\
\Mydiv{2}{1}&1
\end{tabular}

\boxed{\text{0000 0000 0101 1101}}
\end{enumerate}

\item Signed decimal between \(-69\) and \(-192\) inclusive, not divisible by 2, 4, or 8:

-143

\begin{enumerate}
\item Unsigned decimal:

\begin{align*}
15 \times 16^3 &= 61440\\
61440 + 15 \times 16^2 &=  65280\\
65280 + 7\times16 &= 65392\\
65392 + 1 &= \boxed{65393}
\end{align*}
\item Hexadecimal:

Use a hex digit to represent each nybble:

\begin{tabular}{c c c c}
1111 & 1111 & 0111 & 0001\\
F & F & 7 & 1\\
\end{tabular}

\boxed{\text{FF 71}}
\item Binary:

\begin{tabular}{r l}
\Mydiv{2}{143}&1\\
\Mydiv{2}{71}&1\\
\Mydiv{2}{35}&1\\
\Mydiv{2}{17}&1\\
\Mydiv{2}{8}&0\\
\Mydiv{2}{4}&0\\
\Mydiv{2}{2}&0\\
\Mydiv{2}{1}&1\\
\end{tabular}\\

Take two's complement:

\begin{tabular}{c c c c}
0000 & 0000 & 1000 & 1111\\
1111 & 1111 & 0111 & 0001
\end{tabular}

\boxed{\text{1111 1111 0111 0001}}
\end{enumerate}

\item Binary with high-order bits \texttt{0001} and at least 7 \texttt{1}s:

0001 1101 1100 1100

\begin{enumerate}
\item Unsigned decimal:
Signed is positive, unsigned will be same value: \boxed{7628}

\item Signed decimal:

First bit is a 0, will be same as unsigned rep:\\

\begin{align*}
16^3 &= 4096\\
4096 + 13\times16^2 &= 7424\\
7427 + 12\times16 &= 7616\\
7616 + 12 &= \boxed{7628}
\end{align*}

\item Hexadecimal:

\begin{tabular}{c c c c}
0001 & 1101 & 1100 & 1100\\
1 & D & C & C
\end{tabular}

\boxed{\text{1D CC}}

\end{enumerate}

\item Hexadecimal between \texttt{284C} and \texttt{789A} \emph{exclusive}, one zero max:

365B

\begin{enumerate}
\item Unsigned decimal:\\

First bit is a 0, will be same as signed decimal: \boxed{13915}\\

\item Signed decimal:

\begin{align*}
3\times16^3&=12288\\
12288+6\times16^2&=13824\\
13824 + 5\times16&=13904\\
13904 + 11&= \boxed{13915}
\end{align*}

\item Binary:

\begin{tabular}{c c c c}
3&6&5&B\\
0011&0110&0101&1011
\end{tabular}

\boxed{\text{0011 0110 0101 1011}}
\end{enumerate}

\end{enumerate}

\section{Negation}

\emph{Outcome 1a:} Choose 16-bit signed words according to the given specifications, then compute
their negatives, expressing your answers in hex as well. You may have a maximum of eight hex
\texttt{0} digits among your chosen values:

\begin{enumerate}
\item \(x\) in \([\) \texttt{9876}\ldots\texttt{CDEF} \(]  =\) ABCD;
      \(-x = \) 5433

\begin{tabular}{c c c c}
A&B&C&D\\
1010&1011&1100&1101
\end{tabular}      

Flip bits left of rightmost 1:

\begin{tabular}{c c c c}
1010&1011&1100&1101\\
0101&0100&0011&0011
\end{tabular}    

Use a hex digit to represent each nybble:

\begin{tabular}{c c c c}
0101&0100&0011&0011\\
5&4&3&3
\end{tabular}    


\item \(y\) in \([\) \texttt{D219}\ldots\texttt{EDEE} \(]  =\) E3F0;
      \(-y = \) 1C10

\begin{tabular}{c c c c}
E&3&F&0\\
1110&0011&1111&0000
\end{tabular}

Flip all bits left of rightmost 1:

\begin{tabular}{c c c c}
1110&0011&1111&0000\\
0001&1100&0001&0000
\end{tabular}

Use a hex digit to represent each nybble:

\begin{tabular}{c c c c}
0001&1100&0001&0000\\
1&C&1&0
\end{tabular}

\item \(z\) is odd, in \([\) \texttt{8087}\ldots\texttt{9191} \(]  =\) 8F03;
      \(-z = \) 70FD
      
\begin{tabular}{c c c c}
8&F&0&3\\
1000&1111&0000&0011
\end{tabular}\\[2pt]

\begin{tabular}{c c c c}
1000&1111&0000&0011\\
0111&0000&1111&1101
\end{tabular}\\[2pt]

\begin{tabular}{c c c c}
0111&0000&1111&1101\\
7&0&F&D
\end{tabular}\\[2pt]


\item \(w\) is even, in \([\) \texttt{3BB0}\ldots\texttt{5FFE} \(]  =\) 4BDC;
      \(-w = \) B424

\begin{tabular}{c c c c}
4&B&D&C\\
0100&1011&1101&1100
\end{tabular}\\[2pt]

\begin{tabular}{c c c c}
0100&1011&1101&1100\\
1011&0100&0010&0100
\end{tabular}\\[2pt]

\begin{tabular}{c c c c}
1011&0100&0010&0100\\
B&4&2&4
\end{tabular}

\item \(m\) in \([\) \texttt{010A}\ldots\texttt{020B} \(]  =\) 01BF;
      \(-m = \) FE41

\begin{tabular}{c c c c}
0&1&B&F\\
0000&0001&1011&1111
\end{tabular}\\[2pt]

\begin{tabular}{c c c c}
0000&0001&1011&1111\\
1111&1110&0100&0001
\end{tabular}\\[2pt]

\begin{tabular}{c c c c}
1111&1110&0100&0001\\
F&E&4&1
\end{tabular}
\end{enumerate}

\section{Signed Arithmetic}

\emph{Outcome 1a:} Choose 16-bit addends with the given specifications and compute the requested
sums and states using \textbf{signed} arithmetic. You may have a maximum of twelve hex
\texttt{0} digits among your chosen values (not including the ones already given):

\begin{enumerate}
\item \texttt{2 8 C 6} \quad + \quad \texttt{5 A 3 1}
\begin{enumerate}
\item Sum, saturated:\\[2pt]
\begin{tabular}{c c c c}
	{\ \ }2&8&C&6\\
	+5&A&3&1\\
	\hline
	{\ \ }8&2&F&7
\end{tabular}\\[2pt]


Addition of two positives yields a negative result, which indicates overflow.

\boxed{\text{7FFF}}

\item Sum, modular:\\[2pt]

\boxed{\text{82F7}}\\
\item Carry (y/n): N
\item Overflow (y/n): Y
\end{enumerate}

\item \texttt{7 0 F 2} \quad + \quad \texttt{E 7 C D} 
\begin{enumerate}
\item Sum, saturated:\\[2pt]
\begin{tabular}{c c c c c}
&7&0&F&2\\
+&E&7&C&D\\
\hline
1&5&8&B&F
\end{tabular}\\[2pt]

\boxed{\text{58BF}}\\

\item Sum, modular:\\[2pt]
\boxed{\text{58BF}}\\
\item Carry (y/n): Y
\item Overflow (y/n): N
\end{enumerate}

\item \texttt{B B A A} \quad + \quad \texttt{8 A C E}
\begin{enumerate}
\item Sum, saturated:\\
\begin{tabular}{c c c c c}
&B&B&A&A\\
+&8&A&C&E\\
\hline
1&4&7&7&8
\end{tabular}\\

\boxed{\text{8000}}\\
\item Sum, modular: \\
\boxed{\text{4778}}
\item Carry (y/n):Y
\item Overflow (y/n):Y
\end{enumerate}

\item \texttt{A 1 4 B} \quad + \quad \texttt{9 A 7 0}
\begin{enumerate}
\item Sum, saturated:\\
\begin{tabular}{c c c c c}
&A&1&4&B\\
+&9&A&7&0\\
\hline
1&3&B&B&B
\end{tabular}\\[2pt]
\text{Two negatives result in a positive, overflow}\\[2pt]
\boxed{\text{8000}}\\\
\item Sum, modular: \\
\boxed{\text{3BBB}}
\item Carry (y/n):Y
\item Overflow (y/n):Y
\end{enumerate}

\item \texttt{A 2 1 9} \quad + \quad \texttt{2 A D E}
\begin{enumerate}
\item Sum, saturated:\\[2pt]
\begin{tabular}{c c c c c}
&A&2&1&9\\
+&2&A&D&E\\
\hline
&C&C&F&7
\end{tabular}\\[2pt]
\boxed{\text{CCF7}}\\
\item Sum, modular: \boxed{\text{CCF7}}
\item Carry (y/n):N 
\item Overflow (y/n):N
\end{enumerate}

\item \texttt{5 8 0 0} \quad + \quad \texttt{0 F A A}
\begin{enumerate}
\item Sum, saturated: \\
\begin{tabular}{c c c c c}
&5&8&0&0\\
+&0&F&A&A\\
\hline
&6&7&A&A
\end{tabular}\\[2pt]
\boxed{\text{67AA}}\\[2pt]
\item Sum, modular:\\
\boxed{\text{67AA}}\\[2pt]
\item Carry (y/n):N
\item Overflow (y/n):N
\end{enumerate}

\item \texttt{C 0 0 0} \quad + \quad \texttt{C 0 0 0}
\begin{enumerate}
\item Sum, saturated:\\
\begin{tabular}{c c c c c}
&C&0&0&0\\
+&C&0&0&0\\
\hline
1&8&0&0&0
\end{tabular}\\[2pt]
\text{Two negatives, negative result, no overflow}\\
\boxed{\text{8000}}\\[2pt]
\item Sum, modular:\\
\boxed{\text{8000}}\\[2pt]
\item Carry (y/n):Y
\item Overflow (y/n):N
\end{enumerate}

\end{enumerate}

\section{Units of Information}

\emph{Outcome 1a, 6 answers:} Many storage manufacturers sell the same product at different capacities
(e.g., Western Digital My Book; Drobo storage array; SanDisk SDXC Memory Card). Go window shopping and
find product listings for the smallest- and largest-capacity versions of such a product.
\begin{enumerate}
\item (not graded; mainly for reference) Provide the brand, model, min/max capacities, and prices of the
product line you've chosen:

My Book Duo:\\ 
\indent\indent\indent\text{12TB - \$499.99 (promo price)}\\
\indent\indent\indent\text{ 4TB - \$249.99 (promo price)}\\

\item Interpret the device capacities as decimal units (i.e., megabytes, gigabytes, terabytes, etc.).
Show your calculations to answer the following:

\begin{enumerate}
\item How much does a kilobyte cost on the smallest-capacity version of the device?

\begin{align*}
\dfrac{\$249.99}{4\times10^{9}\text{ kb}} = \boxed{\dfrac{\$6.24975\times10^{-8}}{\text{kb}}}
\end{align*}


\item How much does a kilobyte cost on the largest-capacity version of the device?

\begin{align*}
\dfrac{\$499.99}{12\times10^{9}\text{ kb}} = \boxed{\dfrac{\$4.16658\times10^{-8}}{\text{kb}}}
\end{align*}

\item What is the price difference, on a per-kilobyte basis, between the smallest-
and largest-capacity versions of the device?
\begin{align*}
\$6.24975\times10^{-8} - \$4.16658\times10^{-8} = \boxed{\$2.083167\times10^{-8}}
\end{align*}

\end{enumerate}

\item Interpret the device capacities as binary units (i.e., mebibytes, gibibytes, tebibytes, etc.).
Show your calculations to answer the following:

\begin{enumerate}
\item How much does a kibibyte cost on the smallest-capacity version of the device?

\begin{align*}
\dfrac{\$249.99}{4\times2^{30}\text{ kb}} = \boxed{\dfrac{\$5.82053\times10^{-8}}{\text{kb}}}
\end{align*}


\item How much does a kibibyte cost on the largest-capacity version of the device?

\begin{align*}
\dfrac{\$499.99}{12\times2^{30}\text{ kb}} = \boxed{\dfrac{\$3.88043\times10^{-8}}{\text{kb}}}
\end{align*}

\item What is the price difference, on a per-kibibyte basis, between the smallest-
and largest-capacity versions of the device?

\begin{align*}
\$5.82053\times10^{-8} - \$3.88043\times10^{-8} = \boxed{\$1.9401\times10^{-8}}
\end{align*}


\end{enumerate}

\end{enumerate}

\section{IEEE 754 Encoding}

\emph{Outcome 1a, 8 answers:} Read each question carefully and provide the requested answers using the
proper encoding:

\begin{enumerate}
\item Choose a number between 0 and 1 that has at least 4 non-zero digits in the decimal and is
\emph{not} a power of 2 (e.g., \(0.0625\) is \(2^{-4}\) and thus would not count):

\vspace{0.125in}

Your chosen number in decimal form: 0.5625

\begin{enumerate}
\item Single-precision (32-bit) approximation:

Sign is positive, first bit is 0.\\
\begin{align*}
0.5625\times2&=1.125&1\\
0.125\times2&=0.25&0\\
0.25\times2&=0.5&0\\
0.5\times2&=1.0&1\\
&&=0.1001
\end{align*}\\

Normalize:\\
\begin{align*}
0.1001 &= 1.001\times2^{-1}\\
f&=\text{001 0000 0000 0000 0000 0000}\\
\end{align*}
\text{Add bias to exponent of two, place in exponent field:}\\[10pt]
\begin{align*}
2^{8-1} - 1 + (-1) &= 126\\
(126)_{10} &= \text{0111 1110} 
\end{align*}

Put it all together\\[10pt]
\boxed{\text{0 0111 1110 001 0000 0000 0000 0000 0000}}

 

\item Double-precision (64-bit) approximation:

Steps identical until decision of f:\\[10pt]
\begin{align*}
f&=\text{0010 0000 0000 0000 0000 0000 0000 0000 0000 0000 0000 0000 0000}
\end{align*}

\begin{align*}
2^{11-1} - 1 + (-1) = 1022\\
(1022)_{10} = 011 1111 1110
\end{align*}

0011 1111 1110 0010 0000 0000 0000 0000 0000 0000 0000 0000 0000 0000 0000 0000

\boxed{\text{3FE2 0000 0000 0000}}
\end{enumerate}

\item Determine the smallest positive whole number that cannot be represented in memory with the
given floating point encoding, and state why:

\begin{enumerate}
\item \ldots in single-precision (32-bit):

$s = 0,\text{ e = 1111 1110},\text{ f = 111 1111 1111 1111 1111}$

\begin{align*}
(1 + (f \times 2^{-23})) \times 2^{23} &= 1111 1111 1111 1111 1111 1111 \\
1111 1111 1111 1111 1111 1111 + 10 &= \boxed{1\times2^{24} + 1}
\end{align*}

This number cannot be represented in 32-bit IEEE as the least significant bit is too many orders of magnitude away from the most significant bit to be represented in f.

\item \ldots in double-precision (64-bit):

$s = 0,\text{ e = 111 1111 1110},$\\$\text{ f = 1111 1111 1111 1111 1111 1111 1111 1111 1111 1111 1111 1111 1111}$\\


Using the same model as above, :\\

\begin{align*}
(1 + (f \times 2^{-52})) \times 2^{52} &= 1 1111 1111 1111 1111 1111 1111 1111 1111 1111 1111 1111 1111 1111\\
1111 1111 1111 1111 1111 1111 + 10 &= \boxed{1\times2^{53} + 1}
\end{align*}

This number cannot be represented in 64-bit IEEE as the least significant bit is too many orders of magnitude away from the most significant bit to be represented in f.

\end{enumerate}

\item Choose an 8-digit hexadecimal number where \emph{no} digit is repeated nor sequential
(e.g., \texttt{134F EA85} is not allowed because of \texttt{F} followed by \texttt{E}):

\vspace{0.125in}

\texttt{472A 6B31}

\begin{enumerate}
\item Provide the IEEE 754 floating-point value that these bits represent in base 2 (use scientific notation):\\

\text{0100 0111 0010 1010 0110 1011 0011 0001}

$s = 0, \text{ } e = \text{1000 1110}, \text{ } f =\text{010 1010 0110 1011 0011 0001}$

\begin{align*}
e &= 142\\
f \times 2^{-23} + 1 &= \text{1.010 1010 0110 1011 0011 0001}\\
((f \times 2^{-23}) + 1) \times 2^{(142 - 127)}  &= \text{1010 1010 0110 1011 . 0011 0001}\\
&=\boxed{1.010 1010 0110 1011 0011 0001 \times 2^{15}}
\end{align*}


\item Provide its closest approximate value in base 10 (use scientific notation if necessary):\\

\text{Using Dorin's method, which employs elementary school techniques that I now know} \\ \text{need not be shown, get  decimal number left of radix point = 43627}\\

\text{Add to this: }\\
\begin{align*}
43627 + \dfrac{1}{2^{3}} + \dfrac{1}{2^{4}} + \dfrac{1}{2^{8}} &= 43627.1252441797\\
&= \boxed{4.36271252441797 \times 10^4}
\end{align*}


\end{enumerate}

\item Choose 16-digit hexadecimal number where no more than 2 adjacent digits are the same
(e.g., \texttt{0102 EFF3 E157 C411} is not allowed because of \texttt{FF} and \texttt{11}):\\

\texttt{0352 1126 89A2 BC03}

\begin{enumerate}
\item Provide the IEEE 754 floating-point value that these bits represent in base 2 (use scientific notation):\\

Convert to binary:\\

\texttt{0000 0011 0101 0010 0001 0001 0010 0110 1000 1001 1010 0010 1011 1100 0000 0011}\\

$s=0, e=\texttt{000 0011 0101},$\\
$f=\texttt{0010 0001 0001 0010 0110 1000 1001 1010 0010 1011 1100 0000 0011}$\\

$e < 2046, s = 0$ so number is positive:\\

$e = 16 + 32 + 5 = 53$\\
$e - 1023 = -970$\\

$(1 + f\times2^{-52})\times2^{(-970)}  = \boxed{1.00100001 00010010011010001001101000101011110000000011 \times 2^{-970}}$

 

\item Provide its closest approximate value in base 10 (use scientific notation if necessary):

$\boxed{\dfrac{1}{2^{970}} + \dfrac{1}{2^{973}} + \dfrac{1}{2^{978}} + \text{...}}$

\end{enumerate}

\end{enumerate}

\section{Character Encoding}

\emph{Outcome 1b, 44 answers:} Read each question carefully and provide the requested answers using the
proper encoding. \emph{Remember to show your work to prove that you encoded these manually:}

\begin{enumerate}
\item The city of Los Angeles is unhappy with the Unicode SNOWMAN character,
as they are unable to use it on their official documents to represent a fun day in the snow.
They requested a new ``sandman'' symbol in its place.

Pick a codepoint for this new character, and show how it would be encoded in UTF-8, UTF-16, and UTF-32.
The codepoint must be have 6 unique digits and start with a \texttt{10} (i.e., it belongs under
Supplemental Private Use Area-B).

\texttt{103BDE}

\text{Binary rep: 0001 0000 0011 1011 1101 1110}

\begin{enumerate}
\item UTF-8:

Strip leading zeroes and respace digits:

\text{100 000011 101111 011110}\\

Add encoding bits:

\text{11110100 10000011 10101111 10011110}\\

Convert to Hex:

\boxed{\text{F4 83 AF 9E}}

\item UTF-16:

\begin{tabular}{l l l l}
&10&3B&DE\\
-&01&00&00\\
\hline
&0E&3B&DE 
\end{tabular}\\

Convert to binary:

\text{0000 1110 0011 1011 1101 1110}\\

Remove leading zeros and respace bits:

\text{11 10001110 11 11011110}\\

Add in encoding bits:

\text{11011011 10001110 11011111 11011110}\\

Convert to hex:

\boxed{\text{DB 8E DF DE}}\\

\item UTF-32:

\boxed{\text{00 10 3B DE}}

\end{enumerate}

\item Encode the first eight letters of your first and last name combined (including the space
in between) as requested, \emph{replacing four of them} with corresponding characters from the
Enclosed Alphanumerics Unicode block (uppercase or lowercase, your choice):\\

First 8 characters including space: Harris L - Replacing H, both r, and L (

\begin{enumerate}
\item UTF-8:\\

\text{U+24BD U+0061 U+24E1 U+24E1 U+0069 U+0073 U+0020 U+24C1}\\

(H) $\rightarrow$ \text{0010 0100 1011 1101} $\rightarrow$ 1110 0010 1001 0010 1011 1101 $\rightarrow$ E2 92 BD

a $\rightarrow$ \text{0000 0000 0110 0001} $\rightarrow$ 0110 0001 $\rightarrow$ 61

(r) $\rightarrow$ \text{0010 0100 1110 0001} $\rightarrow$ \text{1110 0010 1001 0011 1010 0001} $\rightarrow$ E2 93 A1

i $\rightarrow$ \text{0000 0000 0110 1001} $\rightarrow$ 0110 1001 $\rightarrow$ 69

s $\rightarrow$ \text{0000 0000 0111 0011} $\rightarrow$ 0111 0011 $\rightarrow$ 73

space $\rightarrow$ \text{0000 0000 0010 0000} $\rightarrow$ 0010 0000 $\rightarrow$ 20

(L) $\rightarrow$ \text{0010 0100 1100 0001} $\rightarrow$ 1110 0010 1001 0011 1000 0001 $\rightarrow$ E2 93 81\\

\boxed{\text{E2 92 BD 61 E2 93 A1 E2 93 A1 69 73 20 E2 93 81}}

\item UTF-16:\\

No characters have codepoint > FFFF, so all encodings are still 16 bits\\

\text{U+24BD U+0061 U+24E1 U+24E1 U+0069 U+0073 U+0020 U+24C1}\\

\boxed{\text{24 BD 00 61 24 E1 24 E1 00 69 00 73 00 20 24 C1}}

\item UTF-32:

\boxed{\text{0000 24BD 0000 0061 0000 24E1 0000 24E1 0000 0069 0000 0073 0000 0020 0000 24C1}}

\end{enumerate}

(since we're dealing with Unicode anyway, if your name, when properly written, has an accent or
other diacritical, then use that too)

\item Choose four emoji without variants (the monster master list can be found in
\url{http://unicode.org/emoji/charts/full-emoji-list.html}) to describe your dream vacation.
Encode them:
U+26F0 U+1F3D5 U+1F304 U+1F30C 


\begin{enumerate}
\item UTF-8:

Convert to Binary:\\

\begin{tabular}{l r}
U+26F0&\text{0010 0110 1111 0000}\\
U+1F3D5&\text{0001 1111 0011 1101 0101}\\
U+1F304&\text{0001 1111 0011 0000 0100}\\
U+1F30C&\text{0001 1111 0011 0000 1100}
\end{tabular}\pagebreak

Respace bits and pad:\\

\begin{tabular}{l r}
U+26F0&\text{0010 011011 110000}\\
U+1F3D5&\text{000 011111 001111 010101}\\
U+1F304&\text{000 011111 001100 000100}\\
U+1F30C&\text{000 011111 001100 001100}
\end{tabular}\\

Add encoding bits:\\

\begin{tabular}{l r}
U+26F0&\text{11100010 10011011 10110000}\\
U+1F3D5&\text{11110000 10011111 10001111 10010101}\\
U+1F304&\text{11110000 10011111 10001100 10000100}\\
U+1F30C&\text{11110000 10011111 10001100 10001100}
\end{tabular}\\

Convert back to hex:

\begin{tabular}{l r}
U+26F0&\text{E2 9B B0}\\
U+1F3D5&\text{F0 9F 8F 95}\\
U+1F304&\text{F0 9F 8C 84}\\
U+1F30C&\text{F0 9F 8C 8C}
\end{tabular}\\

\boxed{\texttt{E2 9B B0 F0 9F 8F 95 F0 9F 8C 84 F0 9F 8C 8C}}\\

\item UTF-16: \\

Prepare for Conversion to binary:\\

\begin{tabular}{l r}
U+26F0&\text{26 F0}\\
U+1F3D5&\text{0F3D5}\\
U+1F304&\text{0F304}\\
U+1F30C&\text{0F30C}
\end{tabular}\\


Convert all but first to binary:\\

\begin{tabular}{l r}
U+26F0&\text{26 F0}\\
U+1F3D5&\text{0000 1111 0011 1101 0101}\\
U+1F304&\text{0000 1111 0011 0000 0100}\\
U+1F30C&\text{0000 1111 0011 0000 1100}
\end{tabular}\\

Respace bits:\\

\begin{tabular}{l r}
U+26F0&\text{26 F0}\\
U+1F3D5&\text{00 00111100 11 11010101}\\
U+1F304&\text{00 00111100 11 00000100}\\
U+1F30C&\text{00 00111100 11 00001100}
\end{tabular}\\

Add encoding bits:\\

\begin{tabular}{l r}
U+26F0&\text{26 F0}\\
U+1F3D5&\text{1101 1000 0011 1100 1101 1111 11010101}\\
U+1F304&\text{1101 1000 0011 1100 1101 1111 0000 0100}\\
U+1F30C&\text{1101 1000 0011 1100 1101 1111 0000 1100}
\end{tabular}\\

Convert back to hex:\\

\begin{tabular}{l r}
U+26F0&\text{26 F0}\\
U+1F3D5&\text{D8 3C DF D5}\\
U+1F304&\text{D8 3C DF 04}\\
U+1F30C&\text{D8 3C DF 0A}
\end{tabular}\\

\boxed{\texttt{26 F0 D8 3C DF D5 D8 3C DF 04 D8 3C DF 0A}}\\

\item UTF-32:

\boxed{\texttt{00 00 26 F0 00 01 F3 D5 00 01 F3 04 00 01 F3 0C}}

\end{enumerate}

\emph{Fun tip:} Remember that there are \emph{flag} emoji to represent specific locations.

\item This one is given 5 times the weight: explain why \url{https://xkcd.com/380/} is funny.
(yes, it's funny)
Remember, XKCD comics include a mouseover caption that is an integral part of the strip. \\

This xkcd is funny because the basilisk is still fatal on sight \emph{even in emoji form}. Even funnier, though, is the caption, which says that the eye of the basilisk character has the codepoint U+FDD0 which is actually the unicode "no character" symbol. Ha!

\end{enumerate}

\end{document}
